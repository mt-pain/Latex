\documentclass[platex,dvipdfmx]{jsarticle}			% for platex
% \documentclass[uplatex,dvipdfmx]{jlreq}		% for uplatex
\usepackage{graphicx}
\usepackage{bxtexlogo}

\title{AD変換器VAVE案}

\author{前川達平}
\date{\today}
\begin{document}
\maketitle

\section{目的}
本報告書では、AD変換器の製造コストを削減するためのVAVEを提案することを目的とし、現在使用されている筐体材料の代替提案を行う。本製品の筐体には、耐久性と耐熱性に優れたPARA(ポリアリルアミド)を採用しており、特に我々の製品では50%のグラスファイバーを含有するPARAが使用している。医療機器素材としてのPARAの採用は、その化学的耐性と生体適合性によるものである。一方で、AD変換器の透明部品、特にLEDカバーの光学的透明性が必要な箇所には、PSU(ポリサルフォン)が使用されている。PSUはその優れた透明度と耐久性、耐熱性により、光学部品に理想的な材料である。

しかし、これらの高性能プラスチックはコストが高く、大量生産を行う際に経済性に課題をもたらす。PARAやPSUのような高機能材料の全特性が各使用シナリオで常に必要とされるわけではないため、これらの要件を満たしつつもコスト効率を改善できる材料を探索する必要がある。

本報告書では、医療機器の信頼性と機能性を損なうことなく、コストダウンを達成するために、AD変換器に適した材料選定の基準を定義し、代替材料を提案します。こうして、コストを軽減しつつも、必要な性能基準を保持するソリューションを実現することを目指す。
\section{現行材料の特性}

\section{必要強度}
\section{代替材料の特性}
\section{まとめ}
\section{Cloud LaTeX へようこそ}

Cloud LaTeXは,\LaTeX を使った文書の作成・管理をクラウド上で行えるWebサービスです.
\LaTeX を使うと,複雑な数式
\begin{equation}
\frac{\pi}{2} =
\left( \int_{0}^{\infty} \frac{\sin x}{\sqrt{x}} dx \right)^2 =
\sum_{k=0}^{\infty} \frac{(2k)!}{2^{2k}(k!)^2} \frac{1}{2k+1} =
\prod_{k=1}^{\infty} \frac{4k^2}{4k^2 - 1}
\end{equation}
を含んだ読みやすくきれいな文書作成ができます.

本サービスは,\LaTeX 文書をリアルタイムに保存・コンパイルし,ユーザーアカウント別に管理します.
そのため,本サービスにログインするだけで,どこからでも作業を再開でき,ファイルを持ち歩く必要はありません.
また,様々な \LaTeX テンプレートが用意されているので,手軽に文書を作り始めることができます.
\begin{figure}
\centering
\includegraphics[width=70mm]{figures/Sample.png}
\caption{ここにキャプションを挿入します}
\label{fig:model}
\end{figure}

Cloud LaTeXでは,作成されるPDFそのままのレイアウトで表示するPDFビューモードがあり,コンパイル画面を確認しながら文書を作成することができます(図\ref{fig:model})
日本語では, \pLaTeX / \upLaTeX / \LuaLaTeX でのコンパイルが可能です.
また,日本語や英語文書作成だけでなく,中国語・ハングルに対応した \XeLaTeX のコンパイルも可能です.
ぜひ使ってみてください.
\end{document}